\documentclass{article} 
%\documentclass[twoside]{article} 

\usepackage{times}	
\usepackage{url}        
\usepackage{graphicx}
\usepackage{longtable}

\title{Using the Software for Creating the Lunar Albedo}
\author{Oleg Alexandrov}

\begin{document}

\maketitle

\section{Inputs and outputs}

The albedo generation algorithm takes the following inputs:
\begin{enumerate}
\item A set of DRG images (Digital Raster Graphic images of the Moon in the GeoTIFFf format)
\item A set of DEM images (Digital Elevation Models in the GeoTIFFf format)
\item The Sun and spacecraft position at the moment each DRG image was
  taken. 
\end{enumerate}

\noindent 

The albedo algorithm uses the Sun/spacecraft position and DEM data to
correct for the illumination conditions in the DRG images, thus
computing Moon's true albedo. The output will be a set of
non-overlapping DRG tiles containing the albedo.

In addition to creating albedo images, the software can also be used
for generating image mosaics (see the parameter REFLECTANCE\_TYPE
below). In that case, the DEM images and Sun/spacecraft positions are
not required.

\section{The settings file}

The parameters controlling the albedo generation algorithm are specified in a
settings file (a sample file is shown in the appendix). It has the following entries:

\begin{longtable}{ l p{7cm} }
DRG\_DIR & The path to the directory of DRG images. \\
\\
DEM\_DIR & The path to the directory of DEM images. \\
\\
SUN\_POSITION\_FILE  & The Sun position file.  \\
\\
SPACECRAFT\_POSITION\_FILE & The spacecraft position file.  \\
\\
NUM\_PROCESSES & How many simultaneous processes to start on a given
computing node. This number is best set to the number of available cores on
the node. \\
\\
TILE\_SIZE & Tile size, in degrees (usually 4 by 4 degrees, or 1 by
1).  The output of the algorithm will be a set of image tiles of this size.\\
\\
SIMULATION\_BOX &         The region in which to compute the albedo, in
the format lon\_min : lon\_max : lat\_min : lat\_max. If this
parameter is not provided, the region will be assumed to be the entire Moon. \\
\\
REFLECTANCE\_TYPE &
The reflectance type. Values: 0 -- the reflectance is not
be used, the result of albedo generation is just an image mosaic, 1 --
the Lambertian reflectance model is used, 2 -- the Lunar-Lambertian
reflectance model is used.\\
\\
SHADOW\_THRESH    &         
 Shadow threshold, with values between 0 to 255 (the input DRG images
 have values between 0 to 255). Its use depends on the variable SHADOW\_REMOVAL\_TYPE.
\\
\\
SHADOW\_REMOVAL\_TYPE & Determines how the shadow pixels will be
removed. Values: 0 -- do not remove shadow pixels, 1 -- remove pixels with values below SHADOW\_THRESH, this is the default. Advanced values: 2 -- remove pixels for which $I/(R*T) < t$ \footnote{$I$ is the
image value, $R$ is the Lambertian reflectance, $T$ is the image
exposure, and $t$ is the shadow threshold.}, 3 -- 
analogous to option 2, but use the Lunar-Lambertian reflectance model
instead of the Lambertian model.\\
\\
TR\_CONST & A constant which determines how bright the albeo will be, a larger value will make the albedo 
appear darker. \\
\\
PHASE\_COEFF\_A1  & The coefficient $A_1$ in the expression $e^{-A_1
  \alpha} + A_2$ (here $\alpha$ is the phase angle) which multiplies
the limb darkening factor.\\
\\
PHASE\_COEFF\_A2  & The coefficient $A_2$ in the expression $e^{-A_1 \alpha} + A_2$ above.\\
\\
MAX\_NUM\_ITER     &
Maximum number of albedo update iterations. If it equals 0, the image
exposure, phase coefficients, and the albedo itself will only be
initialized, and no subsequent iterations will take place to improve
the initial values.\\ 
\\
NO\_DEM\_DATA\_VAL &
The value used in the DEM images to denote that no data is available at the
current pixel. This number is normally stored within the DEM images;
only if it absent there will the value specified here be used.\\
\\
USE\_WEIGHTS &
 If to use weights to seamlessly blend the albedo values obtained for individual images (1 means true, and 0 means false).\\
\\
USE\_NORMALIZED\_COST\_FUN &
 If to enforce that the sum of weights over all images add up to 1 at
 each pixel.  If true, all pixels will contribute in equal amount to
 the cost function. This flag makes a difference only if multiple
 albedo iterations are performed (MAX\_NUM\_ITER $>$ 0). Turning this
 on incurs a significant performance hit.\\
\\
POST\_SCALE\_ALBEDO   &
 If after the last iteration to multiply the albedo by $1 + A_2$, the
 value of the phase function when the phase angle $\alpha$ is 0. \\
\\
COMPUTE\_ERRORS   &
 If at the end of the algorithm to compute the albedo error map, showing at each pixel what the error was in computing the albedo at that pixel.\\
\\
% UPDATE\_HEIGHT   &
%  If to update the terrain height based on the computed albedo (this is an experimental feature, not yet fully supported).\\
% \\

\end{longtable}

\section{Sample settings file}

\begin{verbatim}
# Files/directories
DRG_DIR                   DIM_input_2560mpp
DEM_DIR                   DEM_tiles_sub64
SUN_POSITION_FILE         meta/sunpos.txt
SPACECRAFT_POSITION_FILE  meta/spacecraftpos.txt

# Constants
NUM_PROCESSES             8
TILE_SIZE                 4.0
SIMULATION_BOX            -180:180:-180:180
REFLECTANCE_TYPE          2
SHADOW_THRESH             40
SHADOW_REMOVAL_TYPE       1
TR_CONST                  1.6
PHASE_COEFF_A1            1.4
PHASE_COEFF_A2            0.5
MAX_NUM_ITER              0
NO_DEM_DATA_VAL           -32767

# Actions
USE_WEIGHTS               1
USE_NORMALIZED_COST_FUN   0
COMPUTE_ERRORS            0
\end{verbatim}


\section{Requirements for the input}

As mentioned earlier, the inputs to the albedo generation software are a set of DRG images, a set of DEM images, and the Sun and spacecraft position for each DRG image. Here we specify the exact requirements for the data as expected by the software.

\begin{enumerate}

\item The DRG images must be in a directory and stored in the GeoTIFF format. It is assumed that the intensity values for each image pixel are between 0 and 255 (uint8).  The first 11 characters of each image in that directory must be unique, and will be the means by which we will later look up the Sun and spacecraft position for that image. For example, given the image:

DRG\_input\_sub64/AS15-M-0074\_0075-DRG.tif

the lookup key is the 11-character string: AS15-M-0074.

\item The DEM images must be in a directory and stored in the GeoTIFF
format. They may or may not overlap. If the DEM images overlap, the
values from all images at a given pixel will be averaged. It is expected
that the DEM images are either in int16 or float format. If the
images lack the No Data Value, it must be explicitly set in the
settings file using the variable NO\_DEM\_DATA\_VAL.

\item The Sun positions for the DRG images must be provided as a list in
  a file, with one line in the file for each DRG image. The same must
  hold for the spacecraft position. Each line in these lists needs to
  be in the format: 

lookup\_key x y z

where lookup\_key is the 11-character string that identifies the DRG
image, and x, y, and z are the coordinates of the Sun (spacecraft)
 in the Cartesian coordinate system whose origin is the center of the
 Moon. The values must be in kilometers. An example entry for the Sun
 position is 

AS15-M-0074 -1345328.6130903 151849061.51791 689530.60102555

\item The directories containing the DRG and DEM images must be
  user-writeable, as the software will create in those directories a
  list of the GeoTIFF images contained within them, together with the
  coordinates of the corners of each image. These lists are created
  only once and speed up subsequent computations.

\item The amount of memory used by the algortihm is proportional to the tile
size. Generating the albedo at high resolution, such as 10 meters/pixel
requires using small tiles, such as 1 by 1 degree. Otherwise larger tile
sizes can be used, such as 4 by 4 degrees.

\end{enumerate}

\section{Orthoprojecting ISIS cubes}

The albedo algorithm assumes that the input images are in GeoTIFF
format, and that the sun and spacecraft positions are in text files.
This section describes how to obtain this data from ISIS cubes.

The images can be obtained by orthoprojecting the ISIS cubes 
onto the terrain data (a set of DEM tiles) using the script named {\it orthoproject\_cube.sh} in
the directory PhotometryTK/src/tools. The script assumes that the following packages
are installed: Ames Stereo Pipeline, ISIS libraries, and the GDAL
tools. The paths to them can be set in the script. It can be called
as: \\

orthoproject\_cube.sh input.cub input\_isis\_adjust DEM\_dir mpp num\_proc output.tif \\

The arguments passed to it are, respectively, the cube file to
orthoproject, the ISIS adjust file storing the adjusted camera
position, the directory containing the DEM tiles to orthoproject onto, the desired output resolution in meters per pixel, the
number of processors to use, and the name of the output image.

\section{Using the software and the output}

\begin{enumerate}

\item The albedo reconstruction software can be called by using the
  script {\it reconstruct.sh} in the directory PhotometryTK/src/tools. It is
  used as follows:

reconstruct.sh setings.txt labelStr

Here, setings.txt is the settings file, and labelStr is a
string label.

Using this script requires that Vision Workbench and PhotometryTK be
complied (see the installation notes of these two packages for
instructions). The script reconstruct.sh needs to be edited to specify the 
path to these packages.

\item The output albedo will be in the directory
  albedo\_labelStr/albedo. The output will be a set of image tiles, with each
  pixel being uint8 (the same as the input DRG images).

\item The exposure time for each image computed during albedo estimation is
  stored in the directory albedo\_labelStr/exposure. Other subdirectories
 in albedo\_labelStr contain data used for intermediate computations,
 such as the weights assigned to each image, the averaged DEM in each
 tile, etc.

\end{enumerate}


\section{The algorithm}

The albedo is generated in several steps:

\begin{enumerate}
\setcounter{enumi}{-1}
\item Initial setup. % Note we refer to this step below
\item Compute the weights for each DRG image (refer here to the weights).
\item Create the average DEM for each albedo tile based on the input DEM
  images.
\item Compute the sum of weights over all images at each pixel, if the
  variable \\ 
USE\_NORMALIZED\_COST\_FUN is set to 1. 
\item Compute the initial guess for the exposure of each DRG image
  (refer here to the formula).
\item Compute the initial guess for the albedo in each tile (refer here
  to the formula).
\newcounter{saveenum}  
\setcounter{saveenum}{\value{enumi}}
\end{enumerate}

\noindent 
Next, the algorithm will repeat MAX\_NUM\_ITER times the following steps:

\begin{enumerate}
\setcounter{enumi}{\value{saveenum}}
\item Update the exposure for each DRG image (note: we don't do that anymore as the results are bad).
\item Update the phase coefficients (this is done in two steps, first the components of these coefficients are found for each tile, and then the results from all tiles are combined).
\item Update the albedo in each tile.
\setcounter{saveenum}{\value{enumi}}
\end{enumerate}

\noindent Lastly,  if the flag COMPUTE\_ERRORS is set to 1, the algorithm has
another step.

\begin{enumerate}
\setcounter{enumi}{\value{saveenum}}
\item Compute the albedo error, with one value at each
albedo pixel (refer here to the equation having the formula). 
\end{enumerate}

\noindent The algorithm can be distributed over a very large number of
computing nodes, as long as those nodes share storage. In each of the
steps, some data is read from disk, processed, and then saved back to disk,
to be used by subsequent steps. Each of the steps, except step 0 (the
setup) consists of taking a DRG image or a tile, and performing a
calculation. These calculations are independent of each other, and as
such, they can take place in parallel over the specified nodes.

\section{Advanced usage}

The script {\it reconstruct.sh} computes the albedo by performing a
series of calls to the executable
{\it PhotometryTK/build/src/tools/reconstruct}. Advanced users may call this
program directly to execute one of the steps listed in the albedo
algorithm above. 

The {\it reconstruct} executable takes the following options:

\begin{verbatim}
  -s [ --settings-file ] arg     Settings file
  -r [ --results-directory ] arg Results directory
  -f [ --images-list ] arg       The list of images
  -t [ --tiles-list ] arg        The list of albedo tiles
  -i [ --image-file ] arg        Current image or tile
  --initial-setup                Initial setup
  --save-weights                 Save the weights
  --compute-weights-sum          Compute the sum of weights at each pixel
  --compute-shadow               Compute the shadow
  --init-dem                     Initialize the DEM
  --init-exposure                Initialize the exposure times
  --init-albedo                  Initialize the albedo
  --update-exposure              Update the exposure times
  --update-tile-phase-coeffs     Update the phase coefficients per tile
  --update-phase-coeffs          Update the phase coefficients by combining the
                                 results over all tiles
  --update-albedo                Update the albedo
  --update-height                Update the height (shape from shading)
  --compute-errors               Compute the errors in albedo
  --is-last-iter                 Is this the last iteration
  -h [ --help ]                  Display this help message
\end{verbatim}

For example, to initialize the exposure times for a given image, one may
issue the command:

\begin{center}
reconstruct -s settings.txt -r albedo\_run -f albedo\_run/imagesList.txt -t albedo\_run/albedoTilesList.txt --init-exposure -i DRG\_DIR/AS16-M-2961.tif
\end{center}

Each time the {\it reconstruct} executable is run, it echoes the precise
command and options which were used to call it, so these calls can be
inferred by looking at the output of {\it reconstruct.sh}. 

% \bibliography{bibliography}
\end{document}

